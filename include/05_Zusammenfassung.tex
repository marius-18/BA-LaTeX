



Zum Abschluss dieser Arbeit lässt sich feststellen, dass das Ziel 
der Anpassung von \gc{} an bipartite Graphen zur Reduzierung der Laufzeit erreicht werden konnte.
Durch die Parallelisierung der einzelnen \ct{e} wird der Algorithmus derart beschleunigt, dass --- wie
in Kapitel \ref{kap:result} gezeigt --- auf manchen Testinstanzen ein
Speedup von bis zu einem Wert von 17 erreicht wird, was ein äußerst befriedigendes Resultat darstellt. 
Um eine generellere Aussage über den Speedup vom bipartiten \gc{} im Vergleich zum Standard  \gc{}
zu treffen, müsste man jedoch noch deutlich mehr Laufzeitmessungen auf verschiedenen Instanzen
durchführen.
\\

Im Bezug auf die experimentelle Auswertung zur Bestimmung der schnellsten Methode 
einen \ct{} umzusetzen, könnte man noch weitere Untersuchungen vornehmen.
\\

Zum einen werden die Varianten nur auf Instanzen getestet, welche Nachbarschaften mit 
maximal 4 Millionen Knoten enthalten. Je nachdem auf welchen Graphen der Algorithmus angewendet wird, 
könnte es sinnvoll sein auch Instanzen
mit größeren Nachbarschaften zu betrachten. Lassen sich die möglichen Eingabegraphen
für ein spezielles Einsatzgebiet genauer spezifizieren, wäre es sinnvoll, die verschiedenen
Varianten auch nur auf solchen Graphen zu testen. Dann bleibt zu überprüfen, ob die ausgewählte
Variante immer noch die geringste Laufzeit besitzt.
\\

Zum anderen könnte man die Benchmarks zusätzlich noch auf Rechnern mit anderer Hardware ausführen.
Dabei sind vor allem andere Prozessorarchitekturen interessant.
