Zum Abschluss dieser Arbeit lässt sich feststellen, dass das Ziel, einen Algorithmus
zum Randomisieren massiver bipartiter Graphen zu entwickeln, welcher eine bessere Laufzeit aufweist, als der schon bekannte 
\gc{} Algorithmus, erreicht wurde. 



~\\
\\
\\

Dies ist --- wie in Kapitel \ref{kap:result} besprochen --- gelungen




Jedoch könnte man auch noch Verbesserungen vornehmen:


Im Bezug auf die experimentelle Auswertung zur Bestimmung der schnellsten Methode 
einen \ct{} umzusetzen, könnte man noch folgende Verbesserungen untersuchen.
Zum Einen werden die Varianten nur auf Instanzen getestet, welche Nachbarschaften mit 
maximal 4 Millionen Knoten enthalten. Je Nachdem, 


Wir haben gesehen, wie ein \gc{} Tausch und damit auch ein \ct{} aufgebaut ist. 








std in fehlerbalken aber so klein, dass sie nicht sichtbar sind.

Verbessern könnte man noch:
verschiedene benchmarks, größere instanzen, einfach mehr...

auf anderem Rechner Architektur/system testen 

hier auf jeden Fall noch die Python Laufzeiten undso

Laufzeit Vergleiche...
Ausblick?!
