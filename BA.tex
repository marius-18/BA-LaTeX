\documentclass[a4paper, 11pt]{scrreprt}

%%%%%%%%%%%%%%%%%%%%%%%%%%%%%%%%%%%%%%%%%%%%%%%%%%%%%%%%%%%%%%%%%%%%%%%%%%%%
% Some common includes. Add additional includes you need.
%%%%%%%%%%%%%%%%%%%%%%%%%%%%%%%%%%%%%%%%%%%%%%%%%%%%%%%%%%%%%%%%%%%%%%%%%%%%
\RequirePackage[ngerman]{babel}
\RequirePackage[utf8]{inputenc}
\RequirePackage[T1]{fontenc}
\RequirePackage[margin=23mm,bottom=30mm]{geometry}
\RequirePackage{graphicx}
\RequirePackage{amsmath,amsfonts,amssymb,amsthm}
\RequirePackage{listingsutf8}
\RequirePackage{textcomp}
\RequirePackage{tikz}
\RequirePackage{eurosym}
\usetikzlibrary{snakes}

\linespread{1.25}


\usepackage{subfigure}
\usepackage{url}
\usepackage{pdfpages}

\usepackage{algorithm}
\usepackage[noend]{algpseudocode}


\usepackage{cite}
%\usepackage{multirow, tabularx}
%\usepackage{subcaption}


%%%%%%%%%%%%%%%%%%%%%%%%%%%%%%%%%%%%%%%%%%%%%%%%%%%%%%%%%%%%%%%%%%%%%%%%%%%%
% Defines for mathematical notation. Add additional defines as needed.
%%%%%%%%%%%%%%%%%%%%%%%%%%%%%%%%%%%%%%%%%%%%%%%%%%%%%%%%%%%%%%%%%%%%%%%%%%%%
\def\O{\mathcal{O}}
\def\sort{\mathrm{sort}}
\def\scan{\mathrm{scan}}
\def\dist{\mathrm{dist}}

%% Theorem, Lemma undso
\theoremstyle{plain} %Text ist Kursiv
\newtheorem{theorem}{Satz}[chapter]
\newtheorem{lemma}[theorem]{Lemma}
\newtheorem{proposition}[theorem]{Proposition}
\newtheorem{corollary}[theorem]{Korollar}

\theoremstyle{definition} %Text ist \"upright"
\newtheorem{remark}[theorem]{Bemerkung}
\newtheorem{definition}[theorem]{Definition}
\newtheorem{example}[theorem]{Beispiel}


%% wozu das ?! 
%\usepackage{listings}% http://ctan.org/pkg/listings
%\lstset{
%  basicstyle=\ttfamily,
%  mathescape,
%    showspaces=false,
%  showstringspaces=false
%}

%\lstset{
%   basicstyle=\fontsize{10}{13}\selectfont\ttfamily
%}

%\lstset{literate=%
%  {Ö}{{\"O}}1
%  {Ä}{{\"A}}1
%  {Ü}{{\"U}}1
%  {ß}{{\ss}}1
%  {ü}{{\"u}}1
%  {ä}{{\"a}}1
%  {ö}{{\"o}}1
%}

%\newcommand{\tomorrow}{{\AdvanceDate[1]\today}}


%%%%%%%%%%%%%%%%%%%%%%%%%%%%%%%%%%%%%%%%%%%%%%%%%%%%%%%%%%%%%%%%%%%%%%%%%%%%
%%%%% Titel DEFINITION!
%%%%%%%%%%%%%%%%%%%%%%%%%%%%%%%%%%%%%%%%%%%%%%%%%%%%%%%%%%%%%%%%%%%%%%%%%%%%

%\titlehead{\centering Bachelorarbeit mit dem Thema}

%\title{Bachelorarbeit mit dem Thema \\ ~}

%\subtitle{ Implementierung und experimentelle Untersuchung von Parallelem Global-Curveball zur Randomisierung
%Massiver Bipartiter Graphen\\ ~}
%\author{Marius Hagemann \\ 5732742 \\ s2486252@stud.uni-frankfurt.de \\~}
%\date{\tomorrow}
%\publishers{Betreuer: Prof. Dr. Ulrich Meyer \\ ~ \\ Goethe-Universität Frankfurt am Main \\ Fachbereich Informatik }

%\swapnumbers


%%%%%%%%%%%%%%%%%%%%%%%%%%%%%%%%%%%%%%%%%%%%%%%%%%%%%%%%%%%%%%%%%%%%%%%%
%%%%%   DEFINE MAKROS
%%%%%%%%%%%%%%%%%%%%%%%%%%%%%%%%%%%%%%%%%%%%%%%%%%%%%%%%%%%%%%%%%%%%%%%%

\newcommand{\SorSor}{\textbf{SortSort}}
\newcommand{\SorSea}{\textbf{SortSearch}}
\newcommand{\SeaSor}{\textbf{SearchSort}}
\newcommand{\SetSea}{\textbf{SetSearch}}
\newcommand{\SeaSet}{\textbf{SearchSet}}
\newcommand{\USetSea}{\textbf{USetSearch}}
\newcommand{\SeaUSet}{\textbf{SearchUSet}}
\newcommand{\perm}{\textbf{Permutation}}
\newcommand{\distr}{\textbf{Distribution}}
\newcommand{\true}{\textbf{true}}
\newcommand{\false}{\textbf{false}}

\newcommand{\gc}{Global Curveball}
\newcommand{\ct}{Curveball-Tausch}
\newcommand{\cb}{Curveball}

\newcommand{\partvek}{Partitions-Array}


\newcommand{\la}{large}
\newcommand{\sm}{small}
\newcommand{\fr}{fraction}


\newcommand{\nk}{NetworKit }
\newcommand{\cpp}{C++ }

\newcommand{\red}[1]{\textcolor{red}{\textbf{#1}}}
\newcommand{\blue}[1]{\textcolor{blue}{#1}}
\newcommand{\fett}[1]{\textbf{#1}}



\begin{document}
	
%%%%%%%%%%%%%%%%%%%%%%%%%%%%%%%%%%%%%%%%%%%%%%%%%%%%%%%%%%%%%%%%%%%%%%%%
%%%%%   TITELSEITE
%%%%%%%%%%%%%%%%%%%%%%%%%%%%%%%%%%%%%%%%%%%%%%%%%%%%%%%%%%%%%%%%%%%%%%%%

\begin{titlepage}
	\centering
	{\LARGE Bachelorarbeit mit dem Thema\par}
	%\vspace{1.5cm}
	\vfill
	{\huge\bfseries Implementierung und experimentelle Untersuchung 
					von Parallelem Global-Curveball zur Randomisierung
					Massiver Bipartiter Graphen\par}
	\vspace{2.5cm}
	{\Large verfasst von: \par}
	\vspace{1cm}
	{\LARGE \scshape Marius Hagemann\par}
	\vspace{0.5cm}
	{\large Matrikelnummer 5732742 \\ marius@ing-hagemann.de}
	\vfill
	{\Large \today}
	\vfill
	{\LARGE Betreuer:\\ Prof. Dr. Ulrich Meyer \\ Manuel Penschuck\par}
	\vspace{1.5cm}
	{\LARGE Goethe-Universität Frankfurt am Main \\ ~\\ Fachbereich Informatik}
\end{titlepage}

%\maketitle
\cleardoublepage 
\thispagestyle{empty}

\newpage
\section*{}

%%%%%%%%%%%%%%%%%%%%%%%%%%%%%%%%%%%%%%%%%%%%%%%%%%%%%%%%%%%%%%%%%%%%%%%%
%%%%%   ERKLÄRUNG + Inhaltsverzeichnis
%%%%%%%%%%%%%%%%%%%%%%%%%%%%%%%%%%%%%%%%%%%%%%%%%%%%%%%%%%%%%%%%%%%%%%%%

\includepdf[pages=1]{include/erkabschlbachelorinf}

\newpage

\tableofcontents



%%%%%%%%%%%%%%%%%%%%%%%%%%%%%%%%%%%%%%%%%%%%%%%%%%%%%%%%%%%%%%%%%%%%%%%%
%%%%%   ABSTRACT
%%%%%%%%%%%%%%%%%%%%%%%%%%%%%%%%%%%%%%%%%%%%%%%%%%%%%%%%%%%%%%%%%%%%%%%%


%\begin{abstract}
\chapter*{}


~\\

Ziel dieser Bachelorarbeit ist es, einen \red{effizienten} Algorithmus
zur Randomisierung massiver bipartiter Graphen zu entwickeln. Dazu 
wurde das Konzept des \gc{} für bipartite Graphen
angepasst. Es werden verschiedene Möglichkeiten zur Umsetzung diskutiert.
Anhand von Benchmarks wird unter den getesteten Methoden diejenige ausgewählt,
welche die geringste Laufzeit aufweist.
Im Vergleich zu dem schon existierenden \gc{} Algorithmus 
wird mit dem in dieser Arbeit entwickelten Algorithmus 
auf manchen Testinstanzen ein Speedup von bis zu 17 erreicht.



%\end{abstract} 



%%%%%%%%%%%%%%%%%%%%%%%%%%%%%%%%%%%%%%%%%%%%%%%%%%%%%%%%%%%%%%%%%%%%%%%%
%%%%%   EINLEITUNG
%%%%%%%%%%%%%%%%%%%%%%%%%%%%%%%%%%%%%%%%%%%%%%%%%%%%%%%%%%%%%%%%%%%%%%%%


\chapter{Einleitung}
%%%%%%%%%%%%%%%%%%%%%%%%%%%%%%%%%%%%%%%%%%%%%%%%%%%%%%%%%%%%%%%%%%%%%%%%
%%%%%%%% Einleitung
%%%%%%%%%%%%%%%%%%%%%%%%%%%%%%%%%%%%%%%%%%%%%%%%%%%%%%%%%%%%%%%%%%%%%%%%
\glqq Bei der Analyse komplexer Netzwerke, wie beispielsweise soziale Netzwerke, 
werden die zugrundeliegenden Graphen häufig mit zufälligen Graphen verglichen, 
um deren Struktur zu untersuchen.\grqq\footnote{frei übersetzt aus \cite{DBLP:conf/esa/CarstensH0PTW18} Abschnitt 1}
\\

Zum Erzeugen von zufälligen Graphen existieren diverse Modelle wie beispielsweise der Erdos-Renyi-Graph \cite{erdos}
oder der Gilbert-Graph \cite{gilbert}.
\red{Diesen beiden Methoden ...?} 
Dabei weisen jedoch die erstellten Graphen kaum eine Ähnlichkeit zu dem zu analysierenden Netzwerk auf.
Deshalb sind meist Zufallsgraphen gesucht, die zu einem gegebenen Graphen eine identische Gradsequenz
besitzen. Im Zufallsgraph soll also jeder Knoten denselben Grad haben wie im originalen Graph.
\\

Ein Algorithmus, welcher diese Eigenschaft erfüllt, ist beispielsweise Curveball \cite{curveball}.
Hierbei wird ein Graph \glqq randomisiert, indem eine Sequenz an lokalen Modifikationen ausgeführt wird\grqq\footnote{\label{ftn:survey}frei übersetzt aus \cite{penschuck2020recent} Abschnitt 6.4}.
Diese lokalen Modifikationen werden als \ct{} bezeichnet. Bei einem \ct{} werden von zwei zufälligen 
Knoten die disjunkten Nachbarschaften durchgetauscht. Damit bleiben die Knotengrade unverändert.
Um den Graph zu randomisieren, werden einige von diesen \ct{en} hintereinander auf jeweils zufälligen Knoten ausgeführt.
Um sicherzugehen, dass alle Knoten Teil eines \ct{es} waren, werden auf diese Weise
\glqq in Erwartung $\Theta(n\log(n))$  \ct{e} benötigt\grqq \footnote{frei übersetzt aus \cite{DBLP:conf/esa/CarstensH0PTW18} Abschnitt 3.3}.
Um dies zu umgehen, \red{wurde} eine Erweiterung namens \gc{} \cite{DBLP:conf/esa/CarstensH0PTW18} eingeführt. 
Ein \gc{} Tausch ist ein \glqq Super-Schritt\grqq \footref{ftn:survey}, in dem mehrere \ct{e} auf jeweils
unterschiedlichen Knoten nacheinander ausgeführt werden, sodass möglichst jeder Knoten Teil eines solchen Tausches ist. 
Somit werden lediglich $\Theta(n)$ viele \ct{e} benötigt, %um sicherzugehen, 
damit möglichst alle Knoten abgedeckt sind.
%Um so viele Knoten wie möglich durch \ct{e} abzudecken, ist somit nur ein einziger \gc{} Schritt notwendig.
%Somit ist nach nur einem \gc{} Schritt möglichst jeder Knoten Teil eines \ct{es} gewesen.
\\

Ziel dieser Bachelorarbeit ist die Anpassung von \gc{} an bipartite Graphen zur 
Reduzierung der Laufzeit. Zum einen werden gewisse 
Eigenschaften von bipartiten Graphen ausgenutzt, sodass Teile des originalen \gc{} Algorithmus
vereinfacht werden können. Zum anderen ist es möglich, einzelne \ct{e} parallel auszuführen.
Auf diese Weise soll --- wie bereits erwähnt --- eine deutlich geringere Laufzeit im Vergleich zu dem ursprünglichen \gc{}
Algorithmus erreicht werden.
Der neu entwickelte Algorithmus wird unter dem Namen \texttt{BipartiteGlobalCurveball} 
Teil vom Open-Source Projekt \nk{} werden.
\\
\\
\newpage
Zu Beginn dieser Arbeit werden die wichtigsten Begriffe und mathematischen Grundlagen definitiert.
Im Anschluss werden verschiedene Methoden, \gc{} umzusetzen,  aufgezeigt und \red{ unter Einbeziehung 
theoretischer Aspekte miteinander verglichen.} 
Mit Hilfe von Benchmark Tests wird schließlich die Methode ausgewählt, welche die geringste Laufzeit 
aufweist.


%\begin{itemize}
%\item wozu Randomisierung?
%-- Als (zufällige) Eingabe um Algorithmen zu testen?
%-- Zum Analysieren von Netzwerken?
%\red{bla}
%\item es gibt auch andere Methoden um zufällige Graphen zu erstellen (zufällige Kanten zwischen Knoten)
%aber dann bleibt die gewollte Struktur nicht erhalten
%also Global Curveball (, bei dem Kanten getauscht werden)
%\item wir beschränken uns hier nur auf den Spezialfall der bipartiten Graphen wodurch es einfacher wird ...?
%\item wozu GlobalCurveball?
%\item warum macht man das?
%\item-- Zufällige Graphen, wobei die Grade aller Knoten erhalten bleiben
%\item
%Was ist networkit? +  Quelle
%\item
%was heißt massiver graph ? 
%hoher Knotengrad??
%\end{itemize}


%%%%%%%%%%%%%%%%%%%%%%%%%%%%%%%%%%%%%%%%%%%%%%%%%%%%%%%%%%%%%%%%%%%%%%%%
%%%%%   Grundlagen
%%%%%%%%%%%%%%%%%%%%%%%%%%%%%%%%%%%%%%%%%%%%%%%%%%%%%%%%%%%%%%%%%%%%%%%%


\chapter{Grundlagen}


%In diesem Kapitel gehen wir auf die wichtigsten theoretischen Grundlagen, die für diese
%Arbeit benötigt werden, ein.

%%%%%%%%%%%%%%%%%%%%%%%%%%%%%%%%%%%%%%%%%%%%%%%%%%%%%%%%%%%%%%%%%%%%%%%%
%%%%%% Definitionen
%%%%%%%%%%%%%%%%%%%%%%%%%%%%%%%%%%%%%%%%%%%%%%%%%%%%%%%%%%%%%%%%%%%%%%%%

\section{Mathematische Definitionen}
Zu Beginn definieren wir die grundlegenden mathematischen Begriffe. Als wichtigste Grundlage dient 
hierbei das Konstrukt des ungerichteten Graphen.
\begin{definition}[Graph] ~\\
Ein (ungerichteter) \textbf{Graph} $G = (V,E)$ ist ein Tupel bestehend aus einer Knotenmenge $V$ und einer Kantenmenge
 $E$. Eine Kante verbindet zwei Knoten miteinander und ist damit eine Menge, aus zwei Knoten.
 Es gilt $E \subseteq \{ \{u,v\} |\ u,v \in V, u \neq v \}$.  
\end{definition}
\begin{definition}[Multigraph] ~\\
Ein \fett{Multigraph} $G$ ist ein Graph, in dem zwischen zwei Knoten mehrere Kanten existieren 
können. Gibt es zwischen zwei Knoten mehrere Kanten, werden diese als Multikanten bezeichnet.
\end{definition}
\noindent
In dieser Arbeit spielen bipartite Graphen, eine zentrale Rolle.
Bei einem bipartiten Graphen kann man die Knotenmenge in zwei Teilmengen teilen, sodass alle Kanten nur zwischen den 
beiden Mengen verlaufen und nicht innerhalb einer Menge. Formal bedeutet dies:
\begin{definition}[bipartiter Graph] ~\\
Ein Graph $G=(V,E)$ heißt \textbf{bipartit}, wenn es Teilmengen $V_1 \subset V$ und $V_2 \subset V$ gibt, für die 
$V_1 \cup V_2 = V$ und $V_1 \cap V_2 = \emptyset$ gilt,
 sodass für jede Kante $e \in E$ ein $u \in V_1$ und ein $v \in V_2$ existiert, sodass $e = \{u,v\}$ gilt.
Die Knotenmengen $V_1$ und $V_2$ werden auch als Partitionsklassen bezeichnet.
\end{definition}

\noindent
Ein Beispiel für einen bipartiten Graphen ist in Abbildung \ref{fig:beispiel_bipartit} dargestellt. Dabei gilt für die Partitionsklassen:
$V_1 = \{v_1,v_2,v_3\}$ und $V_2 = \{v_{4},v_5,v_6\}$. Man sieht deutlich, dass alle Kanten die Partitionenklassen
$V_1$ und $V_2$ \glqq kreuzen\grqq.

\begin{figure}
%bipartiter graph
\centering
\begin{tikzpicture}
\tikzset{node style/.style={shape=circle,draw=black, inner sep=5pt,}}
                                
\node[node style] at (0, 0)     (1)     {$v_1$};
\node[node style] at (0, -1.5)   (2)     {$v_2$};
\node[node style] at (0, -3)     (3)     {$v_3$};
%\node[node style] at (0, -4.5)   (4)     {$v_4$};

\node[node style] at (3, 0)     (5)     {$v_4$};
\node[node style] at (3, -1.5)   (6)     {$v_5$};
\node[node style] at (3, -3)   (7)     {$v_6$};
%\node[node style] at (3, -4.5)   (8)     {$v_8$};

   
\draw[line width=0.1mm, >=latex]
            (1)     edge[right]    node {} (5)
            (1)     edge[right]    node {} (6)
            (1)     edge[right]    node {} (7)
            (2)     edge[right]    node {} (5)
            (2)     edge[right]    node {} (6)
            (3)     edge[right]    node {} (6)
            %(4)     edge[right]    node {} (5)
            %(4)     edge[right]    node {} (8)
            (3)     edge[right]    node {} (7)
;
\end{tikzpicture}
\caption{Beispiel eines bipartiten Graphen}
\label{fig:beispiel_bipartit}
\end{figure}
\red{überleitung
Für einen \ct{}} ist vor allem der Begriff der Nachbarschaft, genauer der gemeinsamen und disjunkten 
Nachbarschaft, entscheidend.
\begin{definition}[Nachbarschaft]~\\
Ein Knoten $u \in V$ heißt \textbf{benachbart} (oder \fett{adjazent}) zu einem 
anderen Knoten $v \in V$, wenn es eine Kante $\{u,v\} \in E $ gibt. Die Menge $N(u)$ aller adjazenten Knoten
von $u$ nennt man \fett{Nachbarschaft}.
\end{definition}
\begin{definition}[gemeinsame und disjunkte Nachbarschaft]~\\
Die \fett{gemeinsame} Nachbarschaft $N_{c}(u,v)$ zweier Knoten $u$ und $v$ ist die Menge aller Knoten, die sowohl
zu $u$ als auch zu $v$ adjazent sind. In der \fett{disjunkten} Nachbarschaft $N_{d}(u,v)$ von $u$ und $v$ sind dagegen 
alle Knoten die nur zu einem der beiden Knoten adjazent sind. \\
Es gilt also $N_{c}(u,v) = N(u) \cap N(v)$ und $N_{d}(u,v) = \big[N(u) \cup N(v)\big]\setminus \big[N(u) \cap N(v) \big]$.{}
Weiterhin gilt $N_{c}(u,v) \cap N_{d}(u,v) = \emptyset$ und $N_{c}(u,v) \cup N_{d}(u,v) = N(u) \cup N(v)$.
Jeder Knoten $x\in \left[N(u) \cup N(v)\right]$ aus den beiden Nachbarschaften liegt also entweder
in der gemeinsamen oder in der disjunkten Nachbarschaft.
\label{def:common_disjoint}
\end{definition}
\noindent
Dabei bemerken wir, dass in einem bipartiten Graphen zwei Knoten aus einer Partitionsklasse nie
in der gegenseitigen Nachbarschaft liegen können. Diesen Fakt werden wir beim Bipartiten \gc{} ausnutzen.
\blue{Zuletzt} sind noch die Begriffe Knotengrad und Gradsequenz relevant.
\begin{definition}[Knotengrad]~\\
Der \fett{Grad} eines Knotens $v \in V$ wird mit $\deg(v)$ bezeichnet und entspricht der Anzahl
der adjazenten Knoten von $v$. Es gilt also $\deg(v) = |N(v)|$ für alle Knoten $v\in V$.
\end{definition}
\begin{definition}[Gradsequenz]~\\
Die \fett{Gradsequenz} eines Graphen $G = (V,E)$ mit $|V| = n$ Knoten ist gegeben durch das Tupel
$D = (d_{1}, \dots, d_{n})$, wobei $d_{i} = \deg(v_{i})$ dem Grad des Knotens $v_{i}$ entspricht.
\end{definition}
\noindent
Im bipartiten Graph aus Abbildung \ref{fig:beispiel_bipartit} hat beispielsweise
der Knoten $v_{1}$ den Grad $\deg(v_{1}) = 3$. Für die Gradsequenz des Graphen gilt: 
$D = (3,2,2,2,3,2)$.




%%%%%%%%%%%%%%%%%%%%%%%%%%%%%%%%%%%%%%%%%%%%%%%%%%%%%%%%%%%%%%%%%%%%%%%%
%%%%%% NetworKit
%%%%%%%%%%%%%%%%%%%%%%%%%%%%%%%%%%%%%%%%%%%%%%%%%%%%%%%%%%%%%%%%%%%%%%%%
\section{\nk}

\red{\nk \cite{nk}} ist ein Open-Source Projekt, dass es zum Ziel hat, \glqq Werkzeuge für die
Analyse großer Netzwerke, in den Größenordnungen von Tausenden bis Milliarden 
von Kanten, zur Verfügung zu stellen\grqq.
\footnote{\red{aus \cite{nk}}}
\\
Innerhalb von \nk Gibt es einfache Graph Datenstrukturen \red{blabla}
\\
Man kann es mit python nutzen \red{blabla}
\blue{\Large hmmm keine Ahnung hier fehlt noch was\dots}



%%%%%%%%%%%%%%%%%%%%%%%%%%%%%%%%%%%%%%%%%%%%%%%%%%%%%%%%%%%%%%%%%%%%%%%%
%%%%%% Global Curveball
%%%%%%%%%%%%%%%%%%%%%%%%%%%%%%%%%%%%%%%%%%%%%%%%%%%%%%%%%%%%%%%%%%%%%%%%

\section{Global Curveball \red{(auf bipartiten Graphen)}}
%\red{\Huge in der bipartiten verison.....}
\label{sec:global_curveball}
\gc{} ist ein Verfahren zum Randomisieren von Graphen.
Die Aufgabe liegt also darin, bei einer gegebenen Gradsequenz $D$, eine uniform verteilte Stichprobe
aus der Menge aller Graphen mit Gradsequenz $D$ zurückzugeben. Durch das Ausführen von \gc{} 
bleibt also für jeden Knoten $v\in V$ sein Grad $\deg(v)$ erhalten. 
\\
Bei einem bipartiten \gc{} ist der Eingabegraph bipartit. Dadurch entstehen 
Vorteile, die algorithmisch ausgenutzt werden können. Zuerst behandeln wir jedoch einen einzelnen \cb{}, welcher
die Grundlage eines jeden \gc{} darstellt.
%\\
%\\
%\red{\Large
%Ein allgemeiner Ansatz wäre...
%Um dies zu erreichen kann man Kanten Tauschen....
%Was soll da dazu??}
%\\
%\\
\\
\\
\fett{\cb{}} ist  ein Prozess, bei dem Kanten zufällig getauscht werden. Bei einem \ct{} werden
zwei verschiedene Knoten $u$ und $v$ ($u\neq v$) zufällig uniform verteilt ausgewählt, deren Nachbarschaft
zufällig durchmischt wird. Da bei der bipartiten Variante  von \cb{}
die Knoten $u$ und $v$ immer beide aus der gleichen Partitionsklasse gezogen werden, ist sichergestellt, dass
es keine Kante zwischen $u$ und $v$ gibt. Somit sind die beiden Knoten nicht in der jeweils anderen Nachbarschaft
enthalten, es gilt folglich $u\notin N(v)$ und $v\notin N(u)$.
Wird die komplette Nachbarschaft \red{$N(u) \cup N(v)$ (das vielleicht einfach weglassen?)} durchmischt und wieder auf $ N(u)$ und $N(v)$ 
aufgeteilt, könnte es passieren, 
dass dadurch Multikanten entstehen,  
nämlich genau dann, wenn ein Knoten aus der gemeinsamen Nachbarschaft getauscht wird, sodass er danach in 
einer der Nachbarschaften doppelt vorkommt.
Um dies zu vermeiden werden ausschließlich die Knoten aus der disjunkten Nachbarschaft $N_{d}(u,v)$ getauscht.
Ein Beispiel für solch einen Tausch ist in Abbildung \ref{fig:curveball_trade_graph} gegeben.
%
%
%
%%%%% CURVEBALL TRADE auf graph
\begin{figure}
\centering
\begin{tikzpicture}
		\tikzset{node style/.style={shape=circle,draw=black, inner sep=5pt,}}
                                
\node[node style] at (0, -0.75)     (1)     {$v_1$};
\node[node style] at (0, -2.25)   (2)     {$v_2$};


\node[node style, fill=gray] at (3, 0)     (5)     {$v_3$};
\node[node style, fill=yellow] at (3, -1.5)   (6)     {$v_4$};
\node[node style, fill=yellow] at (3, -3)   (7)     {$v_5$};


   
\draw[line width=0.1mm, >=latex]
            (1)     edge[right]    node {} (5)
            (1)     edge[right]    node {} (7)
            (2)     edge[right]    node {} (5)
            (2)     edge[right]    node {} (6)

;
\end{tikzpicture}
\hspace{2cm}
\begin{tikzpicture}
\tikzset{node style/.style={shape=circle,draw=black, inner sep=5pt,}}
                                
\node[node style] at (0, -0.75)     (1)     {$v_1$};
\node[node style] at (0, -2.25)   (2)     {$v_2$};

\node[node style, fill=gray] at (3, 0)     (5)     {$v_3$};
\node[node style, fill=yellow] at (3, -1.5)   (6)     {$v_4$};
\node[node style, fill=yellow] at (3, -3)   (7)     {$v_5$};


\draw[line width=0.1mm, >=latex]
            (1)     edge[right]    node {} (5)
            (1)     edge[right]    node {} (6)
            (2)     edge[right]    node {} (5)
            (2)     edge[right]    node {} (7)

;
\end{tikzpicture}
\caption{Es wird ein \ct{} auf den Knoten $v_{1}$ und $v_{2}$ ausgeführt. 
Für die grau markierte gemeinsame Nachbarschaft gilt $N_{c}(v_{1},v_{2}) = \{v_{3}\}$, 
die disjunkte Nachbarschaft $N_{d}(v_{1},v_{2}) = \{v_{4},v_{5}\}$ ist in gelber Farbe gekennzeichnet.
In diesem Beispiel gibt es nur die zwei gegeben Graphen, die durch Tauschen der disjunkten 
Nachbarschaft entstehen können. Ein \ct{} würde dann jeweils mit Wahrscheinlichkeit 0.5 einen der beiden 
Graphen zurückgeben.}
\label{fig:curveball_trade_graph}
\end{figure}
\\
Ein \fett{Global \ct{}} besteht aus mehreren Curveball-Tauschen, wobei möglichst jeder Knoten aus der \red{Partitionsklasse \Large auch nochmal gucken}
Teil eines Curveball-Tausches sein soll.
%
%
% 
%%%%%% Global Curveball auf partitionsarray
\begin{figure}
\centering
  \begin{tikzpicture}[decoration=brace]
      
      
    %% COMMON FÄRBEN  
    \foreach \x in {0,1,2,3,4,5,6,7,8,9,10}
		{
			\fill [ fill =white, draw =black ]  (\x ,0) rectangle (\x+1 ,1) ;
		};
    
\node[] at (-1.8, 0.5)     (5)     {\partvek};
    
    % untere geschweifte Klammer mit Text darunter:
    \foreach \x in {0,2,4,6,8} 
 		\draw[bend angle=60,bend right,  <->,>=latex, very thick] (\x+0.5,0) to  node[below= 1ex] {\cb{}} (\x+1.5,0) ;
	
	
	\draw[bend angle=60,bend right,  <->,>=latex, very thick, color=white] (10.5,0) to  node[below= 1ex] {\textcolor{black}{kein Tausch}} (11.5,0) ;

	\draw[<-,>=latex, very thick] (10.5,0) to  node[below= 1ex] {} (11.0,-0.5) ;
%\node[] at (11.0, -0.7)     (5)     {kein Tausch};

  \end{tikzpicture}
  \caption{\gc{} auf dem zufällig permutierten \partvek}
  \label{fig:global_curveball_trade_vector}
  
\end{figure}
%
%
%
In Abbildung \ref{fig:global_curveball_trade_vector} ist eine Skizze des \partvek{s} gegeben.
Für einen \gc{} Tausch wird das Array zuerst zufällig permutiert, sodass jedes Element an einer zufälligen
Position steht.
Dann wird jeweils unabhängig ein \ct{} auf den Elementen eins und zwei, drei und vier, usw. ausgeführt.
Durch das zufällige Permutieren des Arrays zu Beginn, wird also jeder der Curveball-Tausche auf zwei zufälligen
Knoten ausgeführt. Hat das \partvek{} eine ungerade Anzahl an Elementen, bleibt am Ende ein Element übrig, 
welches nicht Teil von einem \ct{} ist. 
Hierbei können wir ausnutzen, dass der Eingabegraph bipartit ist. Da es innerhalb der Partitionsklasse \red{\Large hier nochmal gucken..}
keine zwei Knoten gibt, welche durch eine Kante miteinander verbunden sind, 
wird bei einem \ct{} auf beliebigen Knoten $u$ und $v$ 
die Nachbarschaft eines anderen Knotens $x$  aus der gleichen Partitionklasse nicht verändert. Somit
\glqq überschneiden\grqq{} sich die einzelnen Curveball-Tausche nicht. Man kann sie daher zeitgleich, 
also parallel, ausführen. Diese Parallelität führt zu einem Laufzeitvorteil.
\\
Der hauptsächliche Unterschied zwischen \gc{} auf allgemeinen und bipartiten Graphen liegt also darin, 
dass die einzelnen \ct{e} sich gegenseitig nicht beeinflussen und daher vollständig parallel behandelt werden können.
\\
\\
Im vollständigen Randomisierungs-Algorithmus werden schließlich mehrere solcher \gc{} Tausche nacheinander
durchgeführt. Die genaue Anzahl lässt sich beim Aufrufen des Algorithmus durch einen Parameter festlegen.
\\
\\
Um zu beweisen, dass das mehrfache Anwenden vom Bipartiten \gc{} eine uniform verteilte
Stichprobe aller Graphen mit gleicher Gradsequenz erzeugt, kann man das Verfahren als Markov-Kette interpretieren.
Dabei entsprechen die Zustände allen Graphen, welche eine identische Gradsequenz wie der Urspungsgraph haben.
Zwischen zwei Zuständen gibt es genau dann einen Übergang, wenn die beiden Graphen durch einen
\gc{}-Tausch ineinander überführbar sind. \red{Es lässt sich zeigen, } dass diese Markov-Kette
aperiodisch, irreduzibel und symmetrisch ist \cite{penschuck2020recent} \red{welche quelle?}. Ebenso ist die Markov-Kette endlich, da die Anzahl an
Graphen mit gegebener Gradsequenz beschränkt ist. In \cite{....?}\red{welche quelle?} wird bewiesen, dass solche
Markov-Ketten \red{zu einer uniformen Verteilung auf den Zuständen konvergiert.}





%%%%%%%%%%%%%%%%%%%%%%%%%%%%%%%%%%%%%%%%%%%%%%%%%%%%%%%%%%%%%%%%%%%%%%%%
%%%%%% Datenstruktur
%%%%%%%%%%%%%%%%%%%%%%%%%%%%%%%%%%%%%%%%%%%%%%%%%%%%%%%%%%%%%%%%%%%%%%%%

\section{Datenstruktur}
\label{sec:datenstruktur}
In \nk werden Graphen in einer eigenen Datenstruktur gespeichert.
Dabei handelt es sich um eine Art Adjazenzlistendarstellung, bei der für jeden Knoten 
in einem Array die Nachbarn gespeichert sind. Die einzelnen Knoten sind ebenfalls in
einem Array gespeichert. Da wir jedoch ausschließlich auf bipartiten ungerichteten Graphen
arbeiten, können wir eine einfachere Datenstruktur nutzen. Dazu transformieren wird den
den Graph, indem wir lediglich
die Knoten aus einer der beiden Bipartitionsklassen 
in einem Array speichern.
Welche der beiden Partitionsklassen ausgewählt wird, bleibt dem \red{Nutzer} überlassen.
Dabei ist es jedoch sinnvoll, die Klasse mit der geringeren Anzahl an Knoten zu wählen, da auf diese
Weise weniger \ct{e} auszuführen sind.
 Dieses Array werden wir als  \red{\fett{\partvek}} bezeichnen. \red{wenn im folgenden von partition gepsrochen
 wird: diese gemeint}
Für jeden dieser Knoten wird jeweils ein Array erstellt, indem alle adjazenten Knoten gespeichert sind.
Wenn $V_{1}$ die ausgewählte Partitionsklasse ist, dann gibt es also für jeden Knoten $v\in V_{1}$
ein Array, welches die Elemente aus $N(v)$ enthält.
Mit dieser Datenstruktur lassen sich effizient zwei zufällige Knoten der einen Partitionsklasse
auswählen, die (disjunkten und gemeinsamen) Nachbarschaften berechnen und Knoten aus der
disjunkten Nachbarschaft tauschen.
%
%
%
%
%
%
%%%%% CURVEBALL TRADE auf Array
\begin{figure}
\centering
  \begin{tikzpicture}[decoration=brace]
      
      
    %% COMMON FÄRBEN  
    \foreach \x in {0,1,2,3}
		{
			\fill [ fill =gray, draw =black ]  (\x ,0) rectangle (\x+1 ,1) ;
			\fill [ fill =gray, draw =black ]  (\x ,-1.5) rectangle (\x+1 ,-0.5) ;
		};

    %% DISJOINT OBEN FÄRBEN  
    \foreach \x in {4,5,6,7,8,9}
		{
			\fill [ fill =yellow, draw =black ]  (\x ,0) rectangle (\x+1 ,1) ;
		};
		
	%% DISJOINT UNTEN FÄRBEN  
    \foreach \x in {4,5,6,7,8,9,10,11,12}
		{
			\fill [ fill =yellow, draw =black ]  (\x ,-1.5) rectangle (\x+1 ,-0.5) ;
		};
    
\node[] at (-0.8, 0.5)     (5)     {$N(u):$};
\node[] at (-0.8, -1)     (5)     {$N(v):$};
    
    % untere geschweifte Klammer mit Text darunter:
    \draw[decorate, yshift=-1ex] (3.8,-1.5) -- node[below=0.4ex] {$N_{c}(u,v)$} (0.2,-1.5);
    \draw[decorate, yshift=-1ex] (12.8,-1.5) -- node[below=0.4ex] {$N_{d}(u,v)$} (4.2,-1.5);


	\draw[bend left=80, <->,>=latex, very thick] (8.5,0.5) to  node[right= 3ex] {vertauschen} (10.5,-1) ;

  \end{tikzpicture}
  \caption{Skizze eines \cb-Tausches auf den Arrays}
  \label{fig:curveball_trade_vector}
  
\end{figure}
\\
%
%
%
%
Wie ein \ct{} in der Datenstruktur, also den beiden Arrays aussieht, ist 
in Abbildung \ref{fig:curveball_trade_vector} skizziert. Dabei werden zuerst die Elemente
der beiden Vektoren in gemeinsame und disjunkte Nachbarschaft aufgeteilt. Die gemeinsame Nachbarschaft ist
wieder in grau gekennzeichnet, die disjunkte in gelb. Bei einem \ct{} bleiben dann die Elemente 
der gemeinsame Nachbarschaft unverändert, während die Elemente aus der 
disjunkten Nachbarschaft zufällig zwischen den beiden 
Arrays getauscht werden.

%%%%%%%%%%%%%%%%%%%%%%%%%%%%%%%%%%%%%%%%%%%%%%%%%%%%%%%%%%%%%%%%%%%%%%%%
%%%%%% Parallelität
%%%%%%%%%%%%%%%%%%%%%%%%%%%%%%%%%%%%%%%%%%%%%%%%%%%%%%%%%%%%%%%%%%%%%%%%

%\section{Parallelisierung}

%\red{muss hier noch was dazu?!}


%\begin{itemize}
%\item \textbf{Notation}, Graph, bipartitheit, randomisierung, ...? 
%\item theoretische grundlagen
%\item beweis, kanten Tauschen wird uniform verteilt... (markov ??) FDSM sequenz -> algo macht komischen Graph draus -> 
%\item Kanten tauschen, globaler Tausch, zufall
%\item Parallelität (OpenMP)?
%\item networkit (was ist das, wofür braucht man das
%\end{itemize}




%%%%%%%%%%%%%%%%%%%%%%%%%%%%%%%%%%%%%%%%%%%%%%%%%%%%%%%%%%%%%%%%%%%%%%%%
%%%%%   Implementierung
%%%%%%%%%%%%%%%%%%%%%%%%%%%%%%%%%%%%%%%%%%%%%%%%%%%%%%%%%%%%%%%%%%%%%%%%


\chapter{Implementierung eines \ct{es} }
Wie bereits in Kapitel \ref{sec:global_curveball} erwähnt muss die disjunkte Nachbarschaft
der beiden Knoten $u$ und $v$ bekannt sein, um einen \ct{} auf diesen Knoten $u$ und $v$
auszuführen. Der \ct{} besteht dann darin, diese Knoten aus $N_{d}(u,v)$ zu durchmischen.
Deshalb beschäftigen wir uns zuerst damit, wie man die disjunkte Nachbarschaft bestimmt.

%%%%%%%%%%%%%%%%%%%%%%%%%%%%%%%%%%%%%%%%%%%%%%%%%%%%%%%%%%%%%%%%%%%%%%%%
%%%%%% COMMON NEIGHBORS
%%%%%%%%%%%%%%%%%%%%%%%%%%%%%%%%%%%%%%%%%%%%%%%%%%%%%%%%%%%%%%%%%%%%%%%%

\section{Bestimmung der disjunkten Nachbarschaft}
\label{sec:common}
Gesucht sind alle Knoten aus der disjunkten Nachbarschaft der Knoten $u$ und $v$.
Nach der Definition \ref{def:common_disjoint} liegt jeder Knoten aus den Nachbarschaften $N(u)$ und $N(v)$ 
entweder in der disjunkten oder in der gemeinsamen Nachbarschaft. Das Ziel besteht also darin, 
für jeden Knoten aus $N(u) \cup N(v)$ zu entscheiden, ob er zu $N_{d}(u,v)$ oder $N_{c}(u,v)$ gehört.
\\
Die Nachbarschaften $N(u)$ und $N(v)$ liegen wie in Abschnitt \ref{sec:datenstruktur} beschrieben jeweils
in einem Array vor. Der Übersichtlichkeit wegen, werden wir die beiden
Arrays ebenfalls mit $N(u)$ und $N(v)$ bezeichnen. Die Aufgabe ist es also,
 für jedes Element aus den beiden Arrays zu entscheiden,
ob es entweder in beiden Arrays vorkommt, oder nur in einem von den beiden. Dafür 
gibt es verschiedene Möglichkeiten, dies umzusetzen.
\\
\\
Als ersten naiven Ansatz könnte man für jedes Element des Arrays $N(u)$ das gesamte andere 
Array $N(v)$ per linearer Suche nach diesem Element durchsuchen. Hierfür ergibt sich eine Laufzeit von
$\O(|N(u)|\cdot|N(v)|)$. Dies ist aber natürlich nicht sinnvoll ist, da das Array $N(v)$ ziemlich oft 
durchlaufen werden muss und wir 
im Falle von massiven Graphen davon ausgehen können, dass die Arrays (also die Nachbarschaften)
 groß werden. 
\\
Dieses Problem kann man beispielsweise verhindern, indem man beide Arrays aufsteigend sortiert. 
Um nun zu herauszufinden,
welche Werte in beiden Arrays vorkommen, muss man lediglich $u$ und $v$ gleichzeitig linear durchlaufen
und testen, ob die Werte gleich sind, oder nicht. Somit muss man jedes Element der beiden Arrays -- nach dem Sortieren -- 
nur einmal betrachten, was offensichtlich zu einer verbesserten Laufzeit im Vergleich zum naiven
Ansatz führt. Man erhält damit eine Laufzeit von $\O(|N(u)|\cdot \log (|N(u)|)  + |N(v)|\cdot\log(|N(v)|))$. 
Diese Variante wird im Folgenden als \SorSor{} bezeichnet. 
\\
Die Laufzeit hängt dabei im Wesentlichen vom Sortieren ab. Daher führen wir eine Invariante ein, 
die wir \fett{vorsortiert} nennen. Bei dieser Invariante nehmen wir an, dass
die Arrays immer im sortierten Zustand vorliegen. Somit würde bei \SorSor{} das Sortieren wegfallen
und man müsste die beiden Arrays nur noch linear durchlaufen, was zu einer Laufzeit von $\O(|N(u)| + |N(v)|)$
führen würde. 
Es ist jedoch nicht offensichtlich, dass die Invariante zu einer insgesamt besseren Laufzeit eines
\ct{es} führt, da ein \ct{} schließlich auch noch aus dem Durchmischen der disjunkten Nachbarschaft besteht.
Dadurch könnte die Invariante verletzt werden, weshalb man am Ende des \ct{es} nochmal sortieren
muss, um sie wieder aufrecht zu erhalten. Ob die Invariante zu einem Vorteil führt, wird im Kapitel 
\ref{cap:tests} anhand von Laufzeitmessungen geklärt.
\\
\\
Die Variante \SorSor{} lässt sich leicht abwandeln, indem wir nur eines der beiden Arrays sortieren.
Auf diese Weise können wir die Laufzeit des einen Sortiervorgangs sparen. Sortieren
wir das größere Array (ohne Beschränkung der Allgemeinheit $N(u)$), bezeichnen wir die Variante als \SorSea{}. Um zu erkennen, 
welche Elemente zur gemeinsamen und welche zur disjunkten Nachbarschaft gehören, kann man für jedes
Knoten aus $N(v)$ per binärer Suche in logarithmischer Zeit prüfen, ob der Knoten auch in $N(u)$ vorhanden ist, oder nicht.
Damit ergibt sich eine Laufzeit von $\O(|N(u)| \cdot \log(|N(u)|) + |N(v)| \cdot \log(|N(u)|)$. Analog dazu
nennen wir die Variante, in der das kleinere Array sortiert wird, \SeaSor. 
Auch hier könnte die vorsortiert Invariante einen Vorteil bringen.
\\
Eine weitere Methode um viele Werte schnell zu durchsuchen, bietet die Datenstruktur \texttt{set}, welche
eines binären Suchbaums entspricht, beispielsweise eines Rot-Schwarz-Baums.
Dabei wird jedes Element des einen Arrays in den Suchbaum eingefügt. 
Für jedes Element des anderen Arrays kann nun in logarithmischer
Zeit bestimmt werden, ob es im Set und somit auch im ursprünglichen Array vorhanden ist.
\\ 
\red{Ein Vorteil dieser Variante könnte darin liegen...? amortisierte laufzeiten? }
\\
Für diese Möglichkeit gibt es ebenfalls zwei 
analoge Varianten, nämlich \SetSea, bei der die Elemente des größeren Arrays in das Set eingefügt werden
und \SeaSet, bei der das kleinere Array zum Set hinzugefügt wird.
Auch bei den beiden Optionen kann es sinnvoll sein, wenn die vorsortiert Invariante aufrecht erhalten wird. Je
nachdem, wie die interne Implementierung des Suchbaums aussieht, könnte es einen Laufzeitvorteil beim Einfügen
der Werte geben, wenn
diese bereits sortiert sind.
\\
Die letzte Methode, die wir an dieser Stelle betrachten werden,
ist die Verwendung der Datenstruktur \textit{unordered\_set}. Diese ist sehr ähnlich wie Set, mit
dem Unterschied, dass die Werte nicht in geordneter Reihenfolge gespeichert werden, sondern
in einer Hash-Tabelle. Ein Vorteil einer Hash-Tabelle liegt darin, dass das Suchen von Elementen
\red{erwartet in konstanter} Zeit erfolgt.
Ebenfalls gibt es hierbei wieder die Varianten, 
in denen das größere Array in das unordered\_set eingefügt wird (\USetSea) 
oder das kleinere (\SeaUSet).{}
Schließlich werden wir auch bei den letzten beiden Methoden prüfen,
 ob die Invariante eventuell zu einer besseren Laufzeit führt.
\\
\\
Zusammenfassend betrachten wir also insgesamt sieben verschiedene Möglichkeiten um die disjunkte
und gemeinsame Nachbarschaft zu berechnen. Für jede dieser Varianten prüfen wir zusätzlich,
ob die vorsortiert Invariante zu einem Laufzeittechnischen Vorteil führt oder ob es sich
nicht lohnt, diese aufrechtzuerhalten.


%%%%%%%%%%%%%%%%%%%%%%%%%%%%%%%%%%%%%%%%%%%%%%%%%%%%%%%%%%%%%%%%%%%%%%%%
%%%%%% Nachbarn Tauschen
%%%%%%%%%%%%%%%%%%%%%%%%%%%%%%%%%%%%%%%%%%%%%%%%%%%%%%%%%%%%%%%%%%%%%%%%

\section{Tauschen der Nachbarn}
\label{sec:trade}
Im vorherigen Teil wurde beschrieben, wie man die gemeinsame und die disjunkte Nachbarschaft zweier Knoten
$u$ und $v$ bestimmt. Nun beschäftigen wir uns damit, wie man die Knoten der disjunkten Nachbarschaft zufällig tauscht. Als Eingabe 
stehen die Arrays $N_{d}(u,v)$, welches alle Knoten aus der disjunkten Nachbarschaft enthält
und $N_{c}(u,v)$, das die gemeinsamen Nachbarn enthält, zur Verfügung. Weiterhin seien $\text{deg}(u)$ und
$\text{deg}(v)$ die ursprünglichen Knotengrade. Zusätzlich definieren wir uns die Mengen
$D_{u} = N_{d}(u,v) \cap N(u)$, in der die disjunkten Nachbarn von Knoten $u$ liegen, 
und $D_{v} = N_{d}(u,v) \cap N(v)$, in der die disjunkten Nachbarn von $v$ liegen. Die anfänglich
leeren Arrays, in denen die Ausgabe -- also die neuen Nachbarschaften von $u$ und $v$ -- zurückgegeben 
werden soll, bezeichnen wir wieder als $N(u)$ und $N(v)$.
Wir betrachten zwei unterchiedliche Möglichkeiten.
\\
\\
Die erste Idee besteht darin,  
das Array der disjunkten Nachbarschaft zufällig zu permutieren, sodass jedes Element an 
einer zufälligen Position steht. Um nun die beide \glqq neuen\grqq{} Nachbarschaften von $u$ und $v$ zu erstellen,
werden zuerst die Knoten aus der gemeinsamen Nachbarschaft in die leeren Arrays $N(u)$ und $N(v)$ kopiert.
Dann werden die ersten $D_{u}$ Elemente aus dem permutierten Array in $N(u)$ kopiert, die restlichen
in $N(v)$. Somit haben die Nachbarschaften durch den Tausch ihre ursprüngliche Größe nicht verändert, es gilt $|N(u)| = \deg(u)$ und
$|N(v)| = \deg(v)$.
Zur besseren Veranschaulichung ist in Abbildung \ref{fig:trade_shuffle} ein Beispiel zu sehen.
\begin{figure}
\centering
  \begin{tikzpicture}[decoration=brace]
    % Die Grundlinie:
    \draw(0,0)--(10,0);
    \draw(0,1)--(10,1);

    % Striche und Beschriftung in Abständen 0, 2, 4, 6, ...
    \foreach \x/\xtext in {0,1,2,3,4,5,6,7,8,9,10}
      \draw(\x,0)--(\x,1) node[below] {};
      
    % untere geschweifte Klammer mit Text darunter:
    \draw[decorate, yshift=-1ex] (3.8,0) -- node[below=0.4ex] {$D_{u}$} (0.2,0);

    \draw[decorate, yshift=-1ex] (9.8,0) -- node[below=0.4ex] {$D_{v}$} (4.2,0);

\node[] at (-1.2, 0.5)     (5)     {$N_{d}(u,v):$};

  \end{tikzpicture}
  \caption{Beispiel für einen Tausch mit der Variante \perm. Dabei wird das Array der 
  disjunkten Nachbarschaft $N_{d}(u,v)$ zufällig permutiert. Die ersten $D_{u}$ Elemente werden der Nachbarschaft
  von $u$ zugeordnet, die restlichen $D_{v}$ zur Nachbarschaft von $v$. }
  \label{fig:trade_shuffle}
\end{figure}
Bei die diesem Verfahren fällt  jedoch auf, dass einige Elemente beim Permutieren unnötig vertauscht werden.
Für jedes Element ist es eigentlich nur entscheidend, ob es unter den ersten $D_{u}$  liegt (also zum Array
$N(u)$ hinzugefügt wird) oder nicht. Auf welcher Position genau es in 
diesen Bereichen liegt, ist nicht von Relevanz. Ohne Beschränkung der Allgemeinheit gelte 
$N(u) \le N(v)$. Dann kann man
also die Laufzeit dieser Variante verbessern, indem nicht das ganze Array der disjunkten Nachbarschaft zufällig permutiert 
wird, sondern nur die ersten $D_{u}$ \red{vielen} Elemente zufällig gewählt werden.
 Dies setzt die Funktion 
\textit{random\_bipartition\_shuffle} um.
\\
Ein Nachteil dieser Methode besteht jedoch darin, dass durch das zufällige Vertauschen die beiden resultierenden Arrays
$N(u)$ und $N(v)$ nicht mehr unbedingt sortiert sind. Damit wird die im Abschnitt \ref{sec:common} beschriebene
Invariante eventuell verletzt. Möchte man die Invariante aufrecht erhalten, müssen somit die beiden Arrays
in einem letzten Schritt nochmals sortiert werden.
Wir nennen diese Variante \perm.
\\
\\
Die zweite Möglichkeit, die wir betrachten, werden wir als \distr{} bezeichnen.
\\
Die Idee besteht dabei, dass wir über jedes Element des Arrays $N_{d}(u,v)$ iterieren und eine Wahrscheinlichkeit
berechnen, mit
der das Element in die Nachbarschaft von $u$ (beziehungsweise $v$) eingefügt werden soll. Dann wird in einem 
Bernoulli Experiment mit genau dieser Wahrscheinlichkeit ein Zufallsbit gezogen. Je nachdem, welchen
Wert das Zufallsbit hat, wird das Element dann entweder in $N(u)$ oder in $N(v)$ kopiert. Dies wird so lange
wiederholt, bis eines der beiden Arrays seine maximale Kapazität erreicht hat. 
\\
Um die Wahrscheinlichkeit zu berechnen werden am Anfang zwei Variablen $n_v$ und $n_u$ initialisiert, 
welche den Kapazitäten der beiden Arrays $u$ und $v$ entsprechen, wenn die Elemente aus der 
gemeinsamen Nachbarschaft nicht berücksichtigt werden. \red{Es gilt also $n_v = |D_{v}|$ und
$n_u= |D_{u}|$.}
Damit hat das erste Element des Arrays $N_{d}(u,v)$ eine Wahrscheinlichkeit von $p_u = \frac{n_u}{n_u+n_v}$, dem
Array $N(u)$ hinzugefügt zu werden und analog eine Wahrscheinlichkeit $p_v = \frac{n_v}{n_u+n_v}$, um
in $N(v)$ zu gelangen. Offensichtlich gilt $p_u + p_v = 1$. Dann wird mit einer der beiden
Wahrscheinlichkeiten das Bernoulli Experiment durchgeführt, wobei es egal ist, welche Wahrscheinlichkeit
man dazu wählt, da $p_u$ genau die Gegenwahrscheinlichkeit von $p_v$ ist und umgekehrt. 
Wählt man beispielsweise $p_u$ und das Experiment liefert eine eins, dann wird das aktuelle Element
in $N(u)$ kopiert. Dabei
hat sich aber offensichtlich die verbleibende Kapazität des Arrays $N(u)$ verringert. Also muss
der Wert $n_u$ dekrementiert werden. Analoges gilt, falls das Element in die Nachbarschaft von $v$ kopiert wird.
Somit ändern sich nach jeder Iteration die Wahrscheinlichkeiten $p_u$ beziehungsweise $p_v$.
Gilt nach irgendeinem Zeitpunkt entweder $n_u = 0$ oder $n_v = 0$, steht offenbar in einem der Arrays 
keine freie Kapazität mehr zur Verfügung. Somit werden die übrigen Elemente, die noch in $N_{d}(u,v)$ vorhanden sind, 
einfach dem anderen Array hinzugefügt.
\\
\red{Dieses Vorgehen ist auch als Reservoir Sampling\cite{...?} wo? bekannt.}
\\
Ein Vorteil dieser Methode ist, dass die in \ref{sec:common} beschriebene Invariante aufrecht erhalten werden kann.
War das Array der disjunkten Nachbarschaft vor Beginn dieser Methode aufsteigend sortiert,
dann sind auch die bisherigen Elemente der Arrays $N(u)$ und $N(v)$ aufsteigend sortiert, da für jedes
Element nacheinander entschieden wurde, ob es zur Nachbarschaft von $u$ oder von $v$ hinzugefügt wird und dabei die
Reihenfolge der Elemente untereinander nicht verändert wurde.
\\
Zum Schluss müssen noch die die gemeinsamen Nachbarn zu den Arrays $N(u)$ und $N(v)$ hinzugefügt werden.
Möchte man die Invariante aufrecht erhalten, dann sind die beiden Arrays wie beschrieben schon
aufsteigend sortiert. Da auch die Elemente aus $N_{c}(u,v)$ aufsteigend sortiert sind, erhält man
die endgültigen Arrays von $N(u)$ und $N(v)$ durch ein Mergen mit $N_{c}(u,v)$. Soll die Invariante jedoch
nicht aufrecht erhalten werden, reicht es aus, die Elemente aus $N_{c}(u,v)$ jeweils an das Ende der beiden Arrays
zu kopieren.


%%%%%%%%%%%%%%%%%%%%%%%%%%%%%%%%%%%%%%%%%%%%%%%%%%%%%%%%%%%%%%%%%%%%%%%%
%%%%%% Global Curveball Tausch
%%%%%%%%%%%%%%%%%%%%%%%%%%%%%%%%%%%%%%%%%%%%%%%%%%%%%%%%%%%%%%%%%%%%%%%%


\section{\ct}
Wie schon in Abschnitt \ref{sec:global_curveball} beschrieben, 
besteht ein Curveball Tausch auf zwei Knoten $u$ und $v$ daraus, die 
gemeinsame und disjunkte Nachbarschaft der beiden Vektoren zu bestimmen
und schließlich die Knoten aus der disjunkten Nachbarschaft zufällig zu tauschen.
\\
Die verschiedenen Methoden die wir hierfür untersuchen werden 
entstehen durch Kombination aller Varianten, die in \ref{sec:common} und \ref{sec:trade} beschrieben 
werden. Alle diese Möglichkeiten sind in der Tabelle \ref{tab:varianten} zusammengefasst.


\begin{table}
	\centering
	\begin{tabular}{c||c|c||c|c}
		 & \multicolumn{2}{c||}{\distr} & \multicolumn{2}{c}{\perm} \\
		 & \true & \false & \true & \false
		\\ \hline\hline
		\SorSor & & & & \\ \hline\hline
		\SeaSor & & & &\\ \hline\hline
		\SorSea & & & &\\ \hline\hline
		\SeaSet & & & &\\ \hline\hline
		\SetSea & & & &\\ \hline\hline
		\SeaUSet & & & &\\ \hline\hline
		\USetSea& & & &
	\end{tabular}
	\caption{Jedes Feld in der Tabelle entspricht einer Variante für einen \ct{}. Die Werte true 
	und false stehen jeweils dafür, ob die vorsortiert Invariante genutzt wird oder nicht.}
	\label{tab:varianten}
\end{table}



%%%%%%%%%%%%%%%%%%%%%%%%%%%%%%%%%%%%%%%%%%%%%%%%%%%%%%%%%%%%%%%%%%%%%%%%
%%%%%% Ṕseudocode
%%%%%%%%%%%%%%%%%%%%%%%%%%%%%%%%%%%%%%%%%%%%%%%%%%%%%%%%%%%%%%%%%%%%%%%%


\section{Pseudocode}
Wie wir in Kapitel \ref{sec:entscheidung} sehen werden, 
hat die Variante mit den Methoden \SorSor{} und \distr{} und der genutzten vorsortiert
Invariante das beste Laufzeitverhalten. Deswegen gehen wir an dieser Stelle 
noch einmal genauer auf diese beiden Methoden ein, indem wir sie in Pseudocode
beschreiben. In Algorithmus \ref{algo:sortsort} ist \SorSor{} beschrieben.
\red{...noch was dazu schreiben...?}

\begin{algorithm}
  \caption{SortSort}\label{algo:sortsort}
  \begin{algorithmic}[1]
    \Procedure{SortSort}{$u,v$}
	  \State U $ \gets N(u)$ \Comment{vorsortierte Nachbarschaft von u}
	  \State V $ \gets N(v)$ \Comment{vorsortierte Nachbarschaft von v}
	  %\State
	  \State nu $\gets 0$ \Comment{Zähler für U}
	  \State nv $\gets 0$ \Comment{Zähler für V}
	  %\State disjoint $ \gets \emptyset$
	  %\State common $ \gets \emptyset$
      
      \While{(nu < U.\texttt{size()}) and (nv < V.\texttt{size()})}
        \If{U[nu] < V[nv]}
			\State disjoint.\texttt{append(}U[nu]\texttt{)} \Comment{Füge das Element in disjoint ein}
			\State nu ++
			\ElsIf{U[nu] > V[nv]}
				\State disjoint.\texttt{append(}V[nv]\texttt{)} \Comment{Füge das Element in disjoint ein}
				\State nv ++
			\ElsIf{U[nu] == V[nv]}
				\State common.\texttt{append(}U[nu]\texttt{)} \Comment{Füge das Element in common ein}
				\State nu ++
				\State nv ++
        \EndIf
      \EndWhile
      \If{nu $\neq$ U.\texttt{size()}} \Comment{Die restlichen Elemente sind disjunkte Nachbarn}
			\State disjoint.\texttt{append(} U[nu], U[nu+1], \dots\texttt{)}
			\Else{}
			\State disjoint.\texttt{append(} V[nv], V[nv+1], \dots\texttt{)}
      \EndIf
      \State \textbf{return} common, disjoint
   \EndProcedure
  \end{algorithmic}
  \label{algo:sortsort}
\end{algorithm}



\begin{algorithm}
\begin{algorithmic}[1]
\Procedure{Distribution}{common, disjoint}
	\State nu $\gets$ U.\texttt{size()} - common.\texttt{size()} \Comment{Kapazität von $u$}
	\State nv $\gets$ V.\texttt{size()} - common.\texttt{size()} \Comment{Kapazität von $v$}
	\State i = 0
	\While{i < disjoint.\texttt{size()}}
		\State X $\sim \mathcal{B}(\frac{\text{nu}}{\text{nu}+\text{nv}})$  \Comment{ziehe ein Zufallsbit Bernoulli verteilt mit Wahrscheinlichkeit $\frac{nu}{nu+nv}$}

		\If{X==1}
			\State U.\texttt{append(}disjoint[i]\texttt{)} \Comment{Füge das Element in U ein}
			\State i++
			\State nu- - \Comment{aktualisiere die Kapazität}
		\Else{} 
			\State V.\texttt{append(}disjoint[i]\texttt{)}\Comment{Füge das Element in V ein}
			\State i++
			\State nv- -\Comment{aktualisiere die Kapazität}
		\EndIf{}
	\EndWhile{}	
	\State U.\texttt{merge(}common\texttt{)} \Comment{Merge U mit der gemeinsame Nachbarschaft}
	\State V.\texttt{merge(}common\texttt{)} \Comment{Merge V mit der gemeinsame Nachbarschaft}
	\State \textbf{return} U, V
\EndProcedure
\end{algorithmic}
	\caption{Distribution}
	\label{algo:distr}
\end{algorithm}





%\red{bei vorsortiert fällt Zeile \ref{algo:inv1} und \ref{algo:inv2} raus!!}
%\\
%\\
%\red{Distribution}





%%%%%%%%%%%%%%%%%%%%%%%%%%%%%%%%%%%%%%%%%%%%%%%%%%%%%%%%%%%%%%%%%%%%%%%%
%%%%%   Experimentelle Untersuchung
%%%%%%%%%%%%%%%%%%%%%%%%%%%%%%%%%%%%%%%%%%%%%%%%%%%%%%%%%%%%%%%%%%%%%%%%


\chapter{Experimentelle Untersuchung}
\label{cap:tests}
Wie im vorherigen Kapitel beschrieben, existieren verschiedene Varianten, einen \gc{} Tausch 
durchzuführen. Für das Finden der gemeinsamen Nachbarschaft betrachten wir sieben verschiedene Methoden, 
für das Tauschen der Nachbarschaft zwei und weiterhin prüfen wir noch, ob es sinnvoll ist, 
die vorsortiert Invariante zu nutzen oder nicht, was ebenfalls zwei Möglichkeiten entspricht.
Kombiniert man all diese Möglichkeiten erhält man also insgesamt 28 verschiedene Varianten für einen \gc{} 
Tausch.
In diesem Kapitel diskutieren wir, welche der Varianten ausgewählt wurde.
%%%%%%%%%%%%%%%%%%%%%%%%%%%%%%%%%%%%%%%%%%%%%%%%%%%%%%%%%%%%%%%%%%%%%%%%
%%%%%% Aufbau
%%%%%%%%%%%%%%%%%%%%%%%%%%%%%%%%%%%%%%%%%%%%%%%%%%%%%%%%%%%%%%%%%%%%%%%%

\section{Versuchsaufbau}
Um die einzelnen Varianten auf Ihre Laufzeit zu testen, wurde eine Art Versuch aufgebaut.
Dazu wurden alle Methoden in \cpp programmiert. Diese wurden dann auf unterschiedlichen
Instanzen getestet und mittels Google Benchmark \cite{benchmark} wurde die Zeit gemessen, 
die für das Ausführen benötigt wurde.
\\
\\
\red{Google Benchmark ist ein \red{Framework} ...?}
\\
\\
Wie %\red{vorher (kapitel ..?)} 
beschrieben benötigen die Methoden als Eingabe keinen Graph, 
sondern lediglich zwei Vektoren, welche jeweils die Nachbarschaft zweier Knoten repräsentieren. Ohne 
Beschränkung der Allgemeinheit nennen wir den größeren (sofern einer der beiden Vektoren größer ist)
$u$ und den kleineren $v$.
Um möglichst gut zu erkennen, wie sich die verschiedenen Methoden bei unterschiedlichen
Eingaben verhalten, messen wir die Laufzeiten für eine ganze Reihe an Instanzen. 
Um ein gutes Bild zu erhalten, sollten folgende Fälle auf jeden Fall abgedeckt sein:

\begin{itemize}
	\item Beide Vektoren liegen in der gleichen Größenordnung
	
	\item Einer der Vektoren ist wesentlich größer als der andere
	
	\item Der Anteil an gemeinsamen Nachbarn ist groß
	
	\item Der Anteil an gemeinsame Nachbarn  ist klein
\end{itemize}
Um dies zu erreichen, \red{erstellen} wir mehrere Runden, in denen der Vektor $u$ von anfänglich 128
Elementen auf bis zu 4.000.000 Elementen vergrößert wird. Innerhalb jeder Runde werden mehrere Durchläufe 
durchgeführt, bei denen der Vektor $u$ eine Größe zwischen 32 Elementen und der jeweiligen Größe von $v$ hat.
Für jeden dieser Durchgänge werden die beiden Vektoren mit zufälligen, aber paarweise verschiedenen,
Werten befüllt, bis sie die entsprechende Größe haben. 
Dabei haben die Vektoren aber offensichtlich keine Elemente gemeinsam, was dazu führen würde, dass ein \gc{} 
Tausch nichts verändern würde. Um sicherzugehen,
dass die gemeinsamen Nachbarschaft nicht leer ist, müssen somit Elemente des einen Vektors 
in den anderen hineinkopiert werden. Damit die Größe der gemeinsamen Nachbarschaft
variiert wird, werden zuerst 10, dann 25, 50 und 75 Prozent der Elemente kopiert. 
\\
Eine einzelne Test Messung lässt sich somit durch ein Tripel \fett{(\la, \sm, \fr)} beschreiben, wobei
\fett{\la} die Größe von $u$ ist, \fett{\sm} die Größe von $v$ und \fett{\fr} der Anteil der gemeinsamen Elemente.





%%%%%%%%%%%%%%%%%%%%%%%%%%%%%%%%%%%%%%%%%%%%%%%%%%%%%%%%%%%%%%%%%%%%%%%%
%%%%%% Messung
%%%%%%%%%%%%%%%%%%%%%%%%%%%%%%%%%%%%%%%%%%%%%%%%%%%%%%%%%%%%%%%%%%%%%%%%

\section{Messung}
Auf die im vorherigen Abschnitt beschriebene Weise, werden die verschiedenen Instanzen erstellt und
mittels Google Benchmark die Zeit gemessen. Aus Zeitgründen werden jedoch nicht alle Werte 
für \la{} und \sm{} erstellt. Deshalb verdoppeln wir in jedem Schritt die Werte von \la{} und \sm{}
anstatt sie um eins zu inkrementieren. Somit ergeben sich insgesamt 672 Instanzen, auf denen die Laufzeiten der 
einzelnen Methoden gemessen werden. Um eventuelle Messfehler zu minimieren, wird
jeder Durchlauf durch den Google Benchmark Parameter \texttt{benchmark\_repetitions} 5 mal wiederholt.
Mit dem Parameter \texttt{benchmark\_min\_time} wird noch festgelegt, dass jede Variante so oft getestet
wird, bis mindestens eine Gesamtlaufzeit von 0.1 Sekunden \red{erreicht} wird. 
Alle Messungen wurden \red{auf einem Rechner} mit \red{64GB} Arbeitsspeicher und 16 Prozessoren vom Typ Intel(R) Xeon(R) CPU E5-2630 v3 @ 2.40GHz,
welche jeweils 8 Kerne und einen Cache von 20 MB haben, ausgeführt.
Mit dieser Konfiguration hat die Dauer für alle 28 Varianten in Summe ungefähr 19 Stunden betragen.




%%%%%%%%%%%%%%%%%%%%%%%%%%%%%%%%%%%%%%%%%%%%%%%%%%%%%%%%%%%%%%%%%%%%%%%%
%%%%%% Auswertung
%%%%%%%%%%%%%%%%%%%%%%%%%%%%%%%%%%%%%%%%%%%%%%%%%%%%%%%%%%%%%%%%%%%%%%%%

\section{Auswertung}
Die ermittelten Messdaten werden schließlich als \texttt{json}-Datei gespeichert. Mit Hilfe von 
Jupyter Notebook lassen sie sich auswerten. Dabei
handelt es sich um ein \red{Tool}, mit dem man Python-Programme auf einfache Art und Weise erstellen und 
ausführen kann.
Innerhalb von Python werden wir die Bibliotheken \fett{Matplotlib} und \fett{pandas} nutzen, um die Daten
zu analysieren und grafisch aufzuarbeiten.
\begin{figure}
\centering
	\includegraphics[width = 0.5\textwidth]{figures/counting.pdf}
	\caption{Welche Variante ist am häufigsten die schnellste?}
	\label{fig:messung_counting}
\end{figure}
\\
Zuerst betrachten wir für jede Instanz, welche Methode am schnellsten war, also für welche die geringste
Laufzeit gemessen wurde. Interessant sind dann jeweils die Methoden, die häufig am schnellsten waren.
In Abbildung \ref{fig:messung_counting} ist dazu ein Balkendiagramm gegeben.
Dabei sieht man eindeutig, dass die Variante (\SeaUSet, \false, \perm) mit Abstand 
auf den meisten Instanzen die schnellste Laufzeit aller Methoden hat. Die
430 Instanzen, auf denen (\SeaUSet, \false, \perm) die schnellste Methode ist, entsprechen einem Anteil von rund
64\%. Mit 179 \glqq gewonnenen\grqq{} Instanzen folgt die Variante (\SorSor, \true, \distr), was einem Anteil von
27\% entspricht. Zusammen ist somit in etwa 91\% aller getesteter Instanzen eine dieser beiden Methoden
die schnellste gewesen. Daher liegt der Schluss nahe, sich beim Suchen der \glqq besten\grqq{} Variante,
auf diese beiden Methoden zu beschränken. Um nicht fälschlicherweise \glqq gute\grqq{} Methoden auszuschließen
betrachten wir einen weiteren Plot in Abbildung \ref{fig:messung_slowdown}. 
\begin{figure}
\centering
	\includegraphics[width = 0.8\textwidth]{figures/slowdown.pdf}
	\caption{\red{Slowdown im Vergleich zur schnellsten Variante}}
	\label{fig:messung_slowdown}
\end{figure}
Um diesen Plot zu erstellen wurde jede Variante einzeln betrachtet.
Dann wird für jede Instanz bestimmt, welche Variante die kürzeste Laufzeit hat und der Quotient
aus dieser Laufzeit und der Laufzeit der betrachteten Variante berechnet. Dieses Verhältnis wird als \red{Slowdown}
bezeichnet und gibt an, um welchen Faktor die Variante langsamer als die schnellste ist. Die Slowdowns
zu jeder Instanz werden schließlich aufsteigend sortiert und als \red{Kurve} in den Plot eingefügt.
Da die Slowdowns jedoch für jede Variante unabhängig voneinader sortiert werden, geht dadurch 
die Ordnung über die Instanzen verloren. Somit entspricht eine Stelle auf der horizontalen Achse 
nicht für jede Variante der gleichen Instanz.
Dieser Plot legt ebenfalls nahe, sich auf die beiden Varianten (\SeaUSet, \false, \perm) und 
(\SorSor, \true, \distr) zu konzentrieren, da die beiden Kurven am wenigsten stark \red{wachsen} und damit 
die Slowdowns vergleichsweise klein sind. Jedoch lässt sich auch hier nicht bestimmen, welche der beiden
Varianten die bessere ist. Ein Vorteil von (\SeaUSet, \false, \perm) ist, 
dass die Methode auf den meisten Instanzen einen Slowdown
von eins hat -- was wir ja auch schon in Abbildung \ref{fig:messung_counting} gesehen haben --
und damit langsamer anwächst. Ein Nachteil liegt aber darin, dass der Slowdown für manche Instanzen
eine Größe von bis zu 10.0 erreicht, während der Slowdown von  (\SorSor, \true, \distr) durch den maximalen
Wert von 3.5 beschränkt ist (in Abbildung \ref{fig:messung_slowdown} als gestrichelte Linie eingezeichnet).
\\
Dieser Nachteil spiegelt sich ebenfalls in Abbildung \ref{fig:messung_mean} wieder. In diesem Plot
wurden über alle Instanzen für jede Variante jeweils die mittlere Laufzeit bestimmt. Obwohl
es nicht in den meisten Fällen die schnellste Variante ist, hat (\SorSor, \true, \distr) die 
kleinste mittlere Laufzeit mit rund 0.0387 Sekunden. Auf Platz zwei folgt
(\SeaUSet, \false, \perm) mit etwa 0.0640 Sekunden, was schon einer Abweichung von ungefähr 60\% entspricht.
Auch in diesem Plot sieht man deutlich, dass es sich nicht lohnt noch weiter Varianten zu betrachten. Zwar hat 
auch (\SorSor, \true, \perm) eine mittlere Laufzeit, die annähernd so groß ist wie (\SeaUSet, \false, \perm),{}
die aber \red{nicht an die schnellste herankommt}. Alle anderen Varianten haben eine deutlich größere mittlere
Laufzeit.
\begin{figure}
\centering
	\includegraphics[width = \textwidth]{figures/mean.pdf}
	\caption{Mittlere Laufzeiten der Varianten über allen Instanzen}
	\label{fig:messung_mean}
\end{figure}
Abschließend betrachten wir noch die zwei ausgewählten Varianten im direkten Vergleich. Dazu wurden
in Abbildung \ref{fig:messung_small} auf der horizontalen Achse die getesteten Instanzen aufgetragen
und dazu die jeweiligen Laufzeiten von (\SorSor, \true, \distr) und (\SeaUSet, \false, \perm) als Kreuze
eingezeichnet. Die Instanzen sind nach aufsteigenden Werten für \sm{} sortiert. Aus Gründen der Übersichtlichkeit
wird bei  der Beschriftung der Instanzen der Teil \fr{} weggelassen, die Instanzen werden 
nur mit \la{}, \sm{} bezeichnet. Weiterhin wurden der
Übersichtlichkeit wegen die Instanzen aus dem Bereich $32 < \text{\sm}  <32768$ ausgelassen, da sie das gleiche
Bild wie die üblichen Instanzen zeigen.
Man sieht dabei, dass sich die Laufzeiten in den meisten Instanzen nicht so stark unterscheiden.
Dies sind vor allem die Instanzen, bei denen der Unterschied zwischen \la{} und \sm{} nicht so groß ist.
In den Instanzen, in denen sich die Werte für \sm{} und \la{} stark unterscheiden, ist jedoch ein 
deutlicher Vorteil von (\SorSor, \true, \distr) zu erkennen. Auch bei den \glqq großen\grqq{} Instanzen
mit Werten von $\text{\sm{}}> 500.000$ hat diese Variante einen deutlichen Laufzeitvorteil.
Auf der Instanz (4.194.304, 4.194.304, 75) beispielsweise hat (\SeaUSet, \false, \perm) eine Laufzeit
von circa 1.628 Sekunden, während (\SorSor, \true, \distr) nur etwa 0.146 Sekunden benötigt. Der Slowdown
beträgt für diese Instanz also in etwa 11. 
\begin{figure}
\centering
	\includegraphics[width = \textwidth]{figures/small_aufsteigend.pdf}
	\caption{Vergleich der Laufzeiten zweier Varianten auf ausgewählten Instanzen}
	\label{fig:messung_small}
\end{figure}


%%%%%%%%%%%%%%%%%%%%%%%%%%%%%%%%%%%%%%%%%%%%%%%%%%%%%%%%%%%%%%%%%%%%%%%%
%%%%%% Fazit
%%%%%%%%%%%%%%%%%%%%%%%%%%%%%%%%%%%%%%%%%%%%%%%%%%%%%%%%%%%%%%%%%%%%%%%%

\section{\red{Fazit}}
\label{sec:entscheidung}
Abschließend muss an Hand der Messdaten entschieden werden, welche Variante zum Einsatz für
einen \ct{} am Besten geeignet ist.
Zusammenfassend haben wir im vorherigen Abschnitt festgestellt, 
dass nur die Varianten (\SorSor, \true, \distr) und (\SeaUSet, \false, \perm) zur Auswahl stehen.
Während die eine Variante am häufigsten die geringste Laufzeit hat, liegt die andere 
bei der durchschnittlichen Laufzeit weiter vorne. Dies liegt vor allem daran, dass sich 
die Laufzeiten bei den Instanzen auf denen (\SeaUSet, \false, \perm) \glqq gewinnt\grqq{} kaum unterscheiden.
Unter den anderen Instanzen gibt es jedoch welche, bei denen (\SorSor, \true, \distr) bis auf einen
Faktor von circa 11 schneller ist.
\\
Um ein bestmögliches Laufzeitverhalten für einen \ct{} zu erreichen, könnte man auf die Idee kommen, 
beide Varianten zusammen zu nutzen. Man könnte sich eine \red{Heuristik} überlegen, die jeweils angibt, 
auf welcher Instanz man welche Variante verwenden sollte. Somit würde bei jedem \ct{} abhängig von 
der Eingabe, also den Nachbarschaften, entschieden werden, welche der beiden Instanzen man benutzt. 
Das Problem dabei liegt jedoch darin, dass bei diesen Varianten einmal die Vorsortiert-Invariante
genutzt wird und einmal nicht. Dies ist natürlich nicht beides gemeinsam möglich. Entweder man hält die
Nachbarschaften immer sortiert, oder eben nicht. Man muss sich also für eine Möglichkeit der
Invariante entscheiden. Dadurch ändert sich dann aber zwangsläufig  eine der beiden Varianten. 
Entscheidet man sich, die Nachbarschaften sortiert zu halten, würde (\SeaUSet, \false, \perm) zu (\SeaUSet, \true, \perm){}
werden, andernfalls würde sich (\SorSor, \true, \distr) zu (\SorSor, \false, \distr) verändern.
Diese beiden \glqq veränderten \grqq{} Varianten haben aber jeweils deutlich schlechtere Laufzeiten
als die ursprünglichen.
\\
Es ist also nicht sinnvoll möglich, die beiden Varianten miteinander zu kombinieren. Wir müssen uns
also auf eine Variante festlegen. Dies ist die Variante (\SorSor, \true, \distr), da sie im Vergleich zur
Alternative -- wie schon beschrieben --
in kaum einer Instanz eine wesentlich schlechtere Laufzeit hatte, jedoch auf manchen Instanzen wesentlich
bessere Laufzeiten. Außerdem ist es die Variante mit der geringsten mittleren Laufzeit. 
Ein Vorteil dieser Variante neben der Laufzeit liegt noch in der Einfachheit. So werden
lediglich die beiden Arrays sortiert und linear durchlaufen. Es muss keine weitere Datenstruktur
erstellt werden -- wie bei der anderen Variante das \texttt{unordered\_set} -- und damit wird
auch kein zusätzlicher Speicherplatz verbraucht.


%%%%%%%%%%%%%%%%%%%%%%%%%%%%%%%%%%%%%%%%%%%%%%%%%%%%%%%%%%%%%%%%%%%%%%%%
%%%%%% Fazit
%%%%%%%%%%%%%%%%%%%%%%%%%%%%%%%%%%%%%%%%%%%%%%%%%%%%%%%%%%%%%%%%%%%%%%%%

\section{\red{Einordnung?! passen die ergebnisse zu erwartungen?!}}
\SorSor{} konnte man erwarten...
warum der rest so schlecht? invariante ?


%\begin{itemize}

%\item Versuche beschreiben. Aufbau, was warum wie wo?
%\item \red{\Large was wurde erwartet, wie passt das ergebnis dazu? was bedeutet es?}
%\item google Test/ google benchmark
%\item
%bestimmung der schnellsten Varianten..\\
%-- disjoint neighbors\\
%-- trade\\
%-- auf welcher Maschine? gluten\\
%plots\\



%\end{itemize}




%%%%%%%%%%%%%%%%%%%%%%%%%%%%%%%%%%%%%%%%%%%%%%%%%%%%%%%%%%%%%%%%%%%%%%%%
%%%%%   Zusammenfassung
%%%%%%%%%%%%%%%%%%%%%%%%%%%%%%%%%%%%%%%%%%%%%%%%%%%%%%%%%%%%%%%%%%%%%%%%


\chapter{Zusammenfassung}
Zum Abschluss dieser Arbeit





Wir haben gesehen, wie ein \gc{} Tausch und damit auch ein \ct{} aufgebaut ist. 





Verbessern könnte man noch:
verschiedene benchmarks, größere instanzen, einfach mehr...

auf anderem Rechner Architektur/system testen 



Laufzeit Vergleiche...



~\\
\\
\\
\\
\\
was hat das alles gebracht? \\
Algo wird Teil von networkit?\\
ausblick?\\
was könnte man verbessern? mehre tests, andere maschinen
\\
was ist die Laufzeit von einem \gc{} tausch im mittel? für große graphen?

~\\
\\
\\

\red{\Huge TODO:}
\begin{itemize}
	\item Einleitung
	\item Zusammenfassung
	\item \red{\Large \cb{} bezeichnungen undso einheitlich?!}
	\item \red{\Large Anführungszeichen suchen!! + Leerzeichen dahinter}
	\item \red{\huge Laufzeit test für das python ding.. wie schnell funktioniert der algo so?}
	\item \red{wie siehts aus mit der anzahl an global trades?!}
	\item \red{\Large -- Deckblatt?! \\ -- Seitenrand \\ --Zeilenabstand \\ -- Schriftgröße \\ --}
	\item \red{noch was rotes?} \blue{oder was blaues?}
	\item \red{vektor/array?!}
	\item \red{\LaTeX format ?!}
\end{itemize}





%%%%%%%%%%%%%%%%%%%%%%%%%%%%%%%%%%%%%%%%%%%%%%%%%%%%%%%%%%%%%%%%%%%%%%%%
%%%%%   bibliography
%%%%%%%%%%%%%%%%%%%%%%%%%%%%%%%%%%%%%%%%%%%%%%%%%%%%%%%%%%%%%%%%%%%%%%%%


\bibliography{quellen}
\bibliographystyle{plain}


%%%%%%%%%%%%%%%%%%%%%%%%%%%%%%%%%%%%%%%%%%%%%%%%%%%%%%%%%%%%%%%%%%%%%%%%
%%%%%   Abbildungsverzeichnis / Tabellenverzeichnis
%%%%%%%%%%%%%%%%%%%%%%%%%%%%%%%%%%%%%%%%%%%%%%%%%%%%%%%%%%%%%%%%%%%%%%%%


\listoffigures{}
\listoftables{}


\end{document} 
